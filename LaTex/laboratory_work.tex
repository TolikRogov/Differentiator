\documentclass[12pt, letterpaper]{report}
\usepackage{amsthm,amssymb}
\usepackage{mathtext}
\usepackage[T1]{fontenc}
\usepackage[utf8]{inputenc}
\usepackage{pgfplots}
\usepgfplotslibrary{colormaps}
\usepackage{sectsty}
\usepackage[explicit]{titlesec}
\pgfplotsset{compat=1.9}
\usepackage[english,russian]{babel}
\makeatletter
\renewcommand{\@chapapp}{Пункт}
\makeatother
\chaptertitlefont{\Large}
\begin{document}
\begin{titlepage}
\begin{center}
\vspace*{1cm}
\textbf{МОСКОВСКИЙ ФИЗИКО-ТЕХНИЧЕСКИЙ ИНСТИТУТ (НАЦИОНАЛЬНЫЙ ИССЛЕДОВАТЕЛЬСКИЙ УНИВЕРСИТЕТ)}\\
\vspace{0.5cm} Физтех-школа Радиотехники и компьютерных технологий\\
\vspace{5cm} \LARGE{Лабораторная работа 2.10.10\\
Исследование функции}
\vfill
\large{\textbf{Рогов Анатолий Б01-406}} \\
\large \today
\vspace{0.8cm}
\end{center}
\end{titlepage}
\chapter{Исходная функция}
\hfil $f() = $\\
\begin{figure}[h]
\centering
\begin{tikzpicture}
\begin{axis} [
	legend pos = north west,
	xlabel = {$x$},
	ylabel = {$f$},
	width = 300,
	height = 300,
	restrict y to domain=-30:30,
	grid = major,
	enlargelimits=true,
]
\legend{
	$f^{(0)}$
}
\addplot[blue, samples=750]{};
\end{axis}
\end{tikzpicture}
\caption{График функции}
\end{figure}
